\chapter{绪论}

\section{量子反常霍尔效应}

\subsection{霍尔家族的历史}
霍尔效应

\subsection{共价有机框架材料}
共价有机框架材料,是一类具有可设计和可预测结构的结晶多孔有机聚合物,可以通过共价键的有机单元整合构建而成。由于
其周期性骨架、超低密度、高表面积、良好的拓扑结构和多样的功能,共价有机框架材料在化学和材料科学中引起了相当大的关注。
另一方面,二维共价有机框架晶体可以用拓扑方法将结构块的单体组装在刚性骨架中自下而上设计,
因此二维晶体的奇异电子性质可以由单体的前线轨道确定。 在本章中,我们首先揭示了晶体拓扑的平带与晶体单体的前线轨道之间的联系,
并提出了一般的搜索平带规则。此外,具有铁磁共价有机框架材料可以在参杂一定的空穴或者电子浓度时实现,而共价有机框架材料往往是非磁的。
同时考虑floquet的圆偏振光照射,我们可以实现有趣的enantiomorphic anomalous hall effect (EAHE)。这些发现为设计FTB在费米面附近
提供一种新的思路,提供一个新的策略去研究共价有机框架材料自旋电子学。



\section{FTB的紧束缚模型}

\subsection{Kagome 平带}
我们首先介绍Kagome平带。